%%
% This is an Overleaf template for scientific articles and reports
% using the TUM Corporate Desing https://www.tum.de/cd
%
% For further details on how to use the template, take a look at our
% GitLab repository and browse through our test documents
% https://gitlab.lrz.de/latex4ei/tum-templates.
%
% The tumarticle class is based on the KOMA-Script class scrartcl.
% If you need further customization please consult the KOMA-Script guide
% https://ctan.org/pkg/koma-script.
% Additional class options are passed down to the base class.
%
% If you encounter any bugs or undesired behaviour, please raise an issue
% in our GitLab repository
% https://gitlab.lrz.de/latex4ei/tum-templates/issues
% and provide a description and minimal working example of your problem.
%%


\documentclass[
  english,        % define the document language (english, german)
  font=times,     % define main text font (helvet, times, palatino, libertine)
  twocolumn,      % use onecolumn or twocolumn layout
]{tumarticle}


% load additional packages
\usepackage{cite}
\usepackage{hyperref}
\hypersetup{
    colorlinks=true,
    linkcolor=blue,      
    urlcolor=blue,
    citecolor=blue,
}
\usepackage{algorithm}
\usepackage{algpseudocode}
\usepackage{lineno}
\usepackage{lipsum}

\linenumbers
% article metadata
\title{Title of Report}
% \subtitle{Subtitle of the article}

\author
% [affil=1, orcid=1111-222-33333, email=first.author@tum.de]
{Author Name}
\author{Author Name}
% \author[affil=1]{Last Author}

% \affil[mark=1]{\theDepartmentName, \theUniversityName}
% \affil[mark=2]{Another affiliation}

\date{\today}


\begin{document}

\maketitle

\begin{abstract}
  This is a short abstract summarizing the main points of your report.
\end{abstract}


\section{Introduction}

% \lipsum[3-20]

This template should serve as a starting point to detail your report. I expect 3-4 pages per student, i.e., students working alone should hand in a 3-4 page report and student teams 6-8 pages.
You do not need to stick to this exact format, but this can help you structure the report.

In this introduction section, you should introduce the problem, some general technical terms (e.g. a graph, a drawing, matrix, etc.)
and outline the rest of the report. A reader should have a basic idea of your visualizaiton, and a fellow visualization course student should understand the topic at a high level.


You should make use of various \LaTeX commands to cite sources, include figures and tables, and reference things like sections, equations, etc.
Below are some examples you might find useful (see the source code if you are reading the compiled pdf).

When you refer to a article/paper/document, you should include its bibtex entry in references.bib. Then, you can reference it like this~\cite{example}. 



\section{Data abstraction}

Explain where you got your dataset from, its background and context, and then discuss your data abstraction. Someone who gets your data should be able to reproduce the steps.

\section{Task abstraction}

Clearly state what tasks you expect a user to perform with your visualization. Also explain how they relate to the domain situation.

\section{Technical description}

Describe your the different views of your visualization, how you encoded the data, and argue why this is an appropriate encoding.

\subsection{Views}

Description of views.

\subsection{Visual encoding}

Description of visual encoding.

\subsection{Interactivity}

What interactivity is possible in your visualization.

\section{Evaluation}

Argue why your design is a good design.

\subsection{Case Study}

Walk the reader through an example that explains how a user would use your system. 

\section{Conclusion}

End your report with a short conclusion.


\section{Helpful Latex Commands and Tipps}

You can include figures like this: 
\begin{figure}
    \centering
    \includegraphics[width=0.49\linewidth]{figs/sierpinski3d.pdf}
    \caption{Caption}
    \label{fig:example-label}
\end{figure}

And refer to them like this: \autoref{fig:example-label} or by Fig.~\ref{fig:example-label}.

Make use of additional sections and subsections to organize the document. Detail should be at the level that a 
professional could re-implement the technique from your description, without access to the code.
Pseudocode example is included below. 

\begin{algorithm}
  \caption{An algorithm with caption}\label{alg:cap}
  \begin{algorithmic}
  \Procedure{isEven}{$x$}
    \If{$x\mod 2$ is $0$}
        \Return True
    \ElsIf{$x\mod 2$ is $1$}
        \Return False 
    \EndIf
  \EndProcedure
  \end{algorithmic}
  \end{algorithm}


You likely want to format results in tables. Accompanying writeup should include discussion of results. 

\begin{table}
  \centering
  \caption{A table with a caption.}
  \begin{tabular}{ c c c }
  \hline
   cell1 & cell2 & cell3 \\ \hline
   cell4 & cell5 & cell6 \\  \hline
   cell7 & cell8 & cell9  \\  \hline
  \end{tabular}
  \label{tab:tab1}  
\end{table}


\bibliographystyle{plain}
\bibliography{references}

\end{document}
